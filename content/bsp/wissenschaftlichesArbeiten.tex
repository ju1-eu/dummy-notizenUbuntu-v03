%============================
% Copyright Jan Unger, Wuppertal
% erstellt: 22.12.16
% update:   23.06.17
%============================

%============================
% \chapter{Kapitel}
% \subsection{ }\label{ }\index{ }
%============================

\section{Wissenschaftliche Arbeiten
schreiben}\label{wissenschaftliche-arbeiten-schreiben}

Einführung in Anforderungen und Arbeitstechniken

\subsection{Formales}\label{formales}

\subsubsection{Was kennzeichnet eine wissenschaftliche
Arbeit?}\label{was-kennzeichnet-eine-wissenschaftliche-arbeit}

\begin{itemize}% Liste Punkt
\item
  schriftliche und systematische Aufarbeitung eines Themas
\item
  Wiedergabe des akt. Forschungsstands
\item
  Veröffentlichung eigener Ergebnisse
\item
  Einhaltung von Standards bezüglich Arbeitsmethoden und Form der
  Darstellung
\end{itemize}

\subsubsection{Formaler Aufbau: Elemente der wissenschaftlichen
Arbeit}\label{formaler-aufbau-elemente-der-wissenschaftlichen-arbeit}

% Tabelle 
\begin{table}[!htbp] % hier einfügen 
	%============================ 
	\caption[T]{T}	% Caption anpassen! 
	%\label{tab:}	% Referenz anpassen! 
	%============================ 
	\medskip 
	\centering 
	%\setlength{\tabcolsep}{5mm} % Spaltenlänge fest 
  %\rowcolors{1}{orange!25}{} % Farbe
	% auto. Spaltenumbruch 
	%\begin{tabularx}{\textwidth}{XX}
	\begin{tabular} {ll}
		\toprule %% Inhalt 
		%============================
Bezeichnung & Paginierung\\
		\midrule
Deckblatt & keine Seitenangabe\\
Inhaltsverzeichnis & römisch\\
Vorwort & arabisch\\
Einleitung & arabisch\\
Hauptteil & arabisch\\
Schluss & arabisch\\
Literaturangaben & arabisch\\
Anlagen & arabisch\\
Erklärung & ohne\\
		%============================ 
		\bottomrule
	%\end{tabularx} 
	\end{tabular}
\end{table}

\subsubsection{Textteil: Funktionen von Einleitung, Hauptteil und
Schluss}\label{textteil-funktionen-von-einleitung-hauptteil-und-schluss}

\textbf{Einleitung}

\begin{itemize}% Liste Punkt
\item
  Enthält das Thema und die Begründung der Themenwahl
\item
  Formulier das Ziel der Arbeit, Ausgangshypothese
\item
  Erläöutert die Gliederung
\item
  Stellt das Vorgehen oder die Methgode vor
\item
  Definiert zentrale Begriffe
\end{itemize}

\textbf{Haupteil}

\begin{itemize}% Liste Punkt
\item
  zentraler Abschnitt der Arbeit
\item
  Diskussion des Forschungssatndes und Daretllung der eigenen
  Erkenntnisse
\item
  Aufbau sollte nachvollziehbar sein und sich aus der
  Argumentationsführung ergeben
\end{itemize}

\textbf{Schluss}

\begin{itemize}% Liste Punkt
\item
  Fasst die wichtigsten Erkenntnisse aus dem Haupteil zusammen
\item
  Bewertet die Ergebnisse - auch im Kontext des Fachgebiets
\item
  Formuliert noch offene Aspekte des Themas
\item
  Greift Fragestellungen auf, die sch anschließen
\end{itemize}

\subsection{Vorbereitungen}\label{vorbereitungen}

\subsubsection{Zeitplan erstellen: Phasen beim Schreiben einer
Arbeit}\label{zeitplan-erstellen-phasen-beim-schreiben-einer-arbeit}

\textbf{Bearbeitungszeit einteilen}

\begin{itemize}% Liste Punkt
\item
  Zeitplan verschafft Überblick über Zeitbudget
\item
  To-do-Liste als Fahrplan
\end{itemize}

\textbf{Drei Arbeitsphasen}

\begin{itemize}% Liste Punkt
\item
  Inhaltliche und organisatorische Vorarbeiten
\item
  Vorbereitung des Schreibprozesses
\item
  Ausarbeitung und Endkorrektur
\end{itemize}

\textbf{Vorarbeiten}

\begin{itemize}% Liste Punkt
\item
  Themensuche, Konzept erstellen
\item
  Literatursuche, Anlegen einer Stoffsammlung
\item
  Planen und Durchführen empirischer (‚Erfahrung, Erfahrungswissen)
  Untersuchungen
\item
  Gliederungsentwurf, Ausgangsthese
\item
  Formalien und techn. Ausstattung prüfen
\end{itemize}

\textbf{Vorbereitungen für das Schreiben}

\begin{itemize}% Liste Punkt
\item
  Auswahl und Exzerpieren (herausschreiben und zusammenfassen) der
  Quellen
\item
  Auswerten empirischer Daten
\item
  Gliederung erstellen
\item
  Illustrationen notwendig?
\end{itemize}

\textbf{Ausarbeitung und Endkorrektur}

\begin{itemize}% Liste Punkt
\item
  Ausformulierung der einzelnen Gliederungspunkte
\item
  Anpassen der Gliederung
\item
  Korrekturläufe: inhaltlich, formal, Rechtschreibung
\end{itemize}

\subsubsection{Konzept erarbeiten: Erste Auseinandersetzung mit dem
Thema}\label{konzept-erarbeiten-erste-auseinandersetzung-mit-dem-thema}

\textbf{Wissenschaftliche Quellen}

\begin{itemize}% Liste Punkt
\item
  Lexika, Fachwörterbücher, Handbücher
\item
  Überblicksartikel, die sich allgemein mit dem Thema beschäftigen
\item
  Tertiärliteratur
\end{itemize}

\textbf{Erst mal alles aufschreiben}

\begin{itemize}% Liste Punkt
\item
  Brainstroming: Thema klar erfassen
\item
  Eigene Gedanken von Anfang an festhalten
\item
  Erste Gliederung erstellen
\item
  Stichwörter zum thema festhalten: Synonyme, Ober-/Unterbegriffe
\end{itemize}

\subsubsection{Mindmaps: Hilfsmittel zur
Strukturierung}\label{mindmaps-hilfsmittel-zur-strukturierung}

\begin{itemize}% Liste Punkt
\item
  Freemind: \url{http://freemind.sourceforge.net}
\item
  MindManager: \url{https://www.mindjet.com/de/}
\item
  MindView: \url{http://www.matchware.com/ge/products/mindview}
\item
  Xmind: \url{http://www.xmind.net/de/}
\end{itemize}

\textbf{Gedankensnipsel ordnen}

\begin{itemize}% Liste Punkt
\item
  Schlüsselbegriffe zueinander in Beziehung setzen
\item
  Hierachien können deutlich gemacht werden
\end{itemize}

\subsection{Recherchieren}\label{recherchieren}

\subsubsection{\texorpdfstring{Wissenschaftliche Literatur: Was ist
\frqq Wissenschaftlichkeit\flqq\,?}{Wissenschaftliche Literatur: Was ist Wissenschaftlichkeit?}}\label{wissenschaftliche-literatur-was-ist-wissenschaftlichkeit}

\textbf{Wissenschaftlichkeit prüfen}

\begin{itemize}% Liste Punkt
\item
  Autor/Verlag im Wissenschaftsbetrieb bekannt?
\item
  Bibliografische Angaben vollständig?
\item
  Quellenangaben, Literaturangaben vorhanden?
\item
  Klare Trennung zwischen Darstellung und Schlussfolgerung?
\end{itemize}

\textbf{Nicht zitierwürdig sind \ldots{}}

\begin{itemize}% Liste Punkt
\item
  Master-/Bachelor- arbeiten, Mitschriften
\item
  Populärwissenschaftliche Werke (Sachbücher)
\item
  Puplikation unbekannter Herkunft
\item
  Qualität dieser Werke ist nicht ausreichend abgesichert
\end{itemize}

\textbf{Wissenschaft online}

\begin{itemize}% Liste Punkt
\item
  Online-Puplikationen unterliegen denselben Kriterien wie gedruckte
  Bücher!
\item
  Wissenschaftlichkeit einer Internetquelle muss sichergestellt sein
\item
  Indiz für die Wissenschaftlichkeit der Quelle: URL
\end{itemize}

\subsubsection{Literaturarten: Von der Monographie zum
Aufsatz}\label{literaturarten-von-der-monographie-zum-aufsatz}

\textbf{Selbstständige Puplikationen}

\begin{itemize}% Liste Punkt
\item
  Monographien
\item
  Lehrbücher
\item
  Dissertationen
\end{itemize}

\textbf{Unselbstständige Puplikationen}

\begin{itemize}% Liste Punkt
\item
  Herausgeberbände, Sammelbände
\item
  Tagungsbände
\item
  Artikel in wissenschaflichen Zeitschriften
\end{itemize}

\textbf{Graue Literatur}

\begin{itemize}% Liste Punkt
\item
  Herausgebr: Universitäten, Institutionen
\item
  Keine ISBN
\item
  Zitierfähig
\end{itemize}

\textbf{Preprints (Vorabdrucke)}

\textbf{Primär-/Sekundärliteratur}

\begin{itemize}% Liste Punkt
\item
  Primärliteratur: Quellen, die unmittelbar mit dem Thema der Arbeit in
  Zusammenhang stehen
\item
  Sekundärliteratur: Literatur, die sich mit Primärliteratur
  wissenschaflich auseinandersetzt
\item
  Tertiärliteratur: fasst die Sekundärliteratur zusammen
\end{itemize}

\subsubsection{Wo finde ich Literatur? - Die wichtige Rolle von
Bibliotheken}\label{wo-finde-ich-literatur-die-wichtige-rolle-von-bibliotheken}

\textbf{Suche in Literaturverzeichnissen}

\begin{itemize}% Liste Punkt
\item
  Über ein Literaturverzeichnis können neue Quellen erschlossen werden
\item
  Ersetzt nich die systematische Literatursuche
\item
  Nachteil: akt. Literatur nicht berücksichtigt, kritische Gegenposition
  nicht berücksichtigt
\end{itemize}

\textbf{Wissenschaftliche Bibliotheken}

\begin{itemize}% Liste Punkt
\item
  Universitätsbibliotheken
\item
  Lehrstuhl-/Institutsbibliotheken
\item
  Staats- und Landesbibliotheken
\item
  Deutsche Bibliothek \url{http://www.dnb.de}
\end{itemize}

\textbf{Bibliotheksverbünde}

\begin{itemize}% Liste Punkt
\item
  Übersicht über Bibltiotheksverbünde:
  \href{http://www.bibliotheksportal.de/bibliotheken/bibliotheken-in-deutschland/bibliothekslandschaft/bibliotheksverbuende.html\#c3445}{Bibltiotheksverbünde}
\item
  Gemeinsamer Bibliotheksverbund der Länder Bremen, Hamburg,
  Mecklenburg-Vorpommern, Niedersachsen, Sachsen-Anhalt,
  Schleswig-Holstein, Thüringen und der Stiftung Preußischer
  Kulturbesitz: \url{http://www.gbv.de/}
\item
  Kooperativer Bibliotheksverbund Berlin-Brandenburg:
  \url{http://www.kobv.de/}
\item
  Hochschulbibliothekszentrum des Landes Nordrhein-Westfalen (hbz):
  \url{https://www.hbz-nrw.de/}
\item
  Hessisches Bibliotheks- und Informationssystem:
  \url{http://www.hebis.de/}
\item
  Bibliotheksservice-Zentrum Baden-Württemberg:
  \url{http://www.bsz-bw.de/swbverbundsystem/index.html}
\item
  Bibliotheksverbund Bayern: \url{http://www.bib-bvb.de/}
\end{itemize}

\textbf{Wichtige Kataloge}

\begin{itemize}% Liste Punkt
\item
  Karlsruher virtueller Katalog:
  \url{http://www.ubka.uni-karlsruhe.de/kvk.html}
\item
  Deutsche digitale Bibliothek:
  \url{https://www.deutsche-digitale-bibliothek.de/}
\end{itemize}

\textbf{Institutionen}

\begin{itemize}% Liste Punkt
\item
  Statistisches Bundesamt: \url{https://www.destatis.de}
\item
  Statistik Austria:
  \url{https://www.statistik.at/web_de/statistiken/index.html}
\item
  Statistik Schweiz:
  \url{http://www.bfs.admin.ch/bfs/portal/de/index.html}
\end{itemize}

\textbf{Informationsdienste}

\begin{itemize}% Liste Punkt
\item
  Recherchedienste von wiss. Institutionen: z.B. Deutsche
  Nationalbilbliothek \url{http://www.dnb.de/nationalbibliografie}
\item
  Fachinformationssysteme
\item
  Subito: Dokumentenlieferdienst
\end{itemize}

\subsubsection{Wie suche ich nach Literatur? - Die richtigen
Schlüsselbegriffe
wählen}\label{wie-suche-ich-nach-literatur-die-richtigen-schluesselbegriffe-waehlen}

\textbf{Treffende Suchbegriffe finden}

\begin{itemize}% Liste Punkt
\item
  Relevante Stichwörter zum thema suchen: Synonyme, Ober-/Unterbegriffe
\item
  Stichwörter Basis für systematische Literatursuche
\item
  Gute Stichwörter sichern gute Trefferquote
\end{itemize}

\textbf{Bibliothekskataloge einsetzen}

\begin{itemize}% Liste Punkt
\item
  Alphabetischer Katalog
\item
  Sachkatalog: systematischer oder Schlagwortkatalog
\item
  Standortkatalog
\end{itemize}

\textbf{Onlinesuche}

\begin{itemize}% Liste Punkt
\item
  OPAC
\item
  Kreuzkatalog
\end{itemize}

\textbf{Suche eingrenzen}

\begin{itemize}% Liste Punkt
\item
  Erweiterte OPAC-Maske erlaubt Kombination von Suchkriterien
\item
  Zusätzliche Optionen: Trunkierung (Platzhalter) (Bsp. *, =?),
  Operatoren (AND, OR, NOT, \frqq \ldots{}\flqq\,)
\end{itemize}

\textbf{Suche im Bibliotheksbestand}

\begin{itemize}% Liste Punkt
\item
  Präsenzbibliothek: direkte Recherche im Bestand
\item
  Vollständigkeit und Aktualität nicht gegeben
\item
  Nur Monographien und Herausgeberbände
\item
  Keine verstreuten Informationen, z.B. Aufsätze, Beiträge in
  Sammelbänden
\end{itemize}

\textbf{Beschaffung von Literatur}

\begin{itemize}% Liste Punkt
\item
  Kauf oder Ausleihe
\item
  Titel auch digitalisiert verfügbar?
\item
  Lektüre in Präsenzbibliothek
\item
  Dokumentenlieferdienste
\end{itemize}

\subsubsection{Systematische Recherche: Veröffentlichungen systematisch
erfassen}\label{systematische-recherche-veroeffentlichungen-systematisch-erfassen}

\textbf{Wissenschaftliche Suchmaschinen}

\begin{itemize}% Liste Punkt
\item
  Suche erfolgt nur in wissenschaftlichen Quellen und Portalen
\item
  Bielefeld Acedemic Search Engine (BASE):
  \url{http://www.base-search.net/de/index.php}
\item
  Forschungsportal.Net: \url{http://forschungsportal.net/}
\item
  Forschungsportal Schweiz: \url{http://www.forschungsportal.ch/}
\item
  Google Scholar: \url{https://scholar.google.de/}
\item
  MetaGer: \url{https://www.metager.de/}
\item
  OAIster: \url{http://www.oclc.org/oaister.en.html}
\end{itemize}

\textbf{Bibliographien}

\begin{itemize}% Liste Punkt
\item
  Unverzichtbar für lückenlose wissenschaftliche Recherche
\item
  Allgemeinbibliographie - Fachbibliographie
\item
  Deutsche Nationalbibliographie
\item
  Zeitschriftendatenbank ZDB
\end{itemize}

\subsubsection{Literaturverwaltungstools - Quellen und Zitate immer im
Griff}\label{literaturverwaltungstools-quellen-und-zitate-immer-im-griff}

\begin{itemize}% Liste Punkt
\item
  Citavi: \url{https://www.citavi.com/}
\item
  Colwiz: \url{https://www.colwiz.com/home}
\item
  Docear: \url{http://www.docear.org/}
\item
  EndNote: \url{http://endnote.com/}
\item
  JabRef: \url{http://jabref.sourceforge.net/}
\item
  Medeley: \url{https://www.mendeley.com/}
\item
  RefWorks: 
\item
  Zotero: \url{https://www.zotero.org/}
\end{itemize}

\subsection{Der Schreibprozess}\label{der-schreibprozess}

\subsubsection{Stoffsammlung auswerten}\label{stoffsammlung-auswerten}

\textbf{Auswahl aus Literaturliste treffen}

\begin{itemize}% Liste Punkt
\item
  Bücher nach bestimmten Stichwörtern durchsuchen oder querlesen
\item
  Kriterien:
\item
  Autor, Titel, Verlag, Jahr
\item
  Inhaltsverzeichnis, Klappentexte
\item
  Einleitung, Abstract
\item
  Personen- oder Sachregister
\item
  Rezensionen
\end{itemize}

\textbf{Lesen und Exzerpieren}

\begin{itemize}% Liste Punkt
\item
  Ausgewählte Literatur erfassen und aneignen
\item
  Exzerpieren: zentrale Stellen oder Herleitungen notieren
\item
  Eigene Gedanken schriftlich festhalten
\item
  Methoden: Karteikarten, Word oder Latex, Literaturverwaltungsprogramm
\end{itemize}

\subsubsection{Gliederung erstellen: Eine nachvollziehbare Struktur
anlegen}\label{gliederung-erstellen-eine-nachvollziehbare-struktur-anlegen}

\textbf{Gliederung als Prozess}

\begin{itemize}% Liste Punkt
\item
  Überschriften knapp und aussagefräftig formulieren
\item
  Klare inhaltiche Gliederung mit Blick auf die zentrale Aussage
\end{itemize}

\textbf{Inhaltliche Gliederungsmodelle}

\begin{itemize}% Liste Punkt
\item
  Grobgliederung abhängig von der Richtung der Argumentation
\item
  Modelle:
\item
  Begründung einer Hypothese
\item
  Vom Ganzen zum Einzelnen
\item
  Argumente bauen aufeinander auf
\item
  Erläuterung einer Entwicklung
\item
  Erläuterung eines Gegensatzes
\end{itemize}

\subsubsection{Arbeit formulieren: Was ist wissenschaftlicher
Stil?}\label{arbeit-formulieren-was-ist-wissenschaftlicher-stil}

\textbf{Wissenschafliches Schreiben}

\begin{itemize}% Liste Punkt
\item
  Eingehende Darstellung
\item
  Wissenschaftlicher Stil hat zwei Aspekte:
\item
  Beschreibung eines sachverhalts
\item
  Begründen von Ergebnissen und Schlussfolgerungen
\item
  Trennung ist Gebot wissenschaflicher Objektivität
\item
  Erkenntnisse müssen nachvollziehbar sein
\end{itemize}

\textbf{Verständlichkeit ist geboten}

\begin{itemize}% Liste Punkt
\item
  Ziel: gut lesbarer Text
\item
  Eindeutige Formulierungen und anschauliche Beschreibungen
\item
  Sätze klar strukturiert
\item
  Zusammenfassungen am Kapitelanfang und -ende
\end{itemize}

\textbf{Fachvokabular verwenden}

\begin{itemize}% Liste Punkt
\item
  Fachtermini gehören zur wissenschaftlichen Ausdrucksweise
\item
  Konsequente und eindeutige Verwendung ist unabdingbar
\item
  Im Zweifel: Begriffserklärung
\end{itemize}

\textbf{Alternative zur \frqq Ich\flqq\, Formulierung}

\begin{itemize}% Liste Punkt
\item
  Dies macht ersichtlich \ldots{}
\item
  Dies macht deutlich \ldots{}
\item
  Aus diesem Sachverhalt geht hervor \ldots{}
\item
  Daraus ergibt sich \ldots{}
\item
  Es fällt auf \ldots{}
\end{itemize}

\subsection{Richtig zitieren}\label{richtig-zitieren}

\subsubsection{Was ist ein Zitat? - Grundsätze wissenschaftlichen
Arbeitens}\label{was-ist-ein-zitat-grundsaetze-wissenschaftlichen-arbeitens}

\textbf{Wann ist ein Zitat sinnvoll?}

\begin{itemize}% Liste Punkt
\item
  Das Zitat enthält zentrale Aussage, mit der sich der eigene Text
  intensiv auseinandersetzt
\item
  Inhalt des Zitats muss erläutert werden:
\item
  Bestätigung der eigenen Position
\item
  Kritische Diskussion der Aussage
\item
  Grundlagen der eigenen Arbeit werden offen gelegt
\end{itemize}

\textbf{Zitatformen}

\begin{itemize}% Liste Punkt
\item
  Direktes Zitat: Wiedergabe des exakten Wortlauts
\item
  Indirektes Zitat: Paraphrasierung
\item
  Zitate sollten sorgfältig ausgewählt sein und dürfen nicht aus dem
  Zusammenhang gerissen werden
\end{itemize}

\subsubsection{Richtiges Zitieren: Korrekte Wiedergabe von
Zitaten}\label{richtiges-zitieren-korrekte-wiedergabe-von-zitaten}

\begin{itemize}% Liste Punkt
\item
  Wortgenaue Wiedergabe von direkten Zitaten
\item
  Auslassungen durch drei Punkte gekennzeichnet: {[}\ldots{}{]}
\item
  Hervorhebungen: {[}Hervorhebung{]}
\item
  Schreibfehler im Original
\end{itemize}

\textbf{Zitatnachweis im Fließtext}

Quellenangabe steht in Klammern im Fließtext: z.B. (Schell 1972, S.14)

\textbf{Zitatnachweis in Fußnoten}

Fußnotenziffer im Fließtext, Quellenangabe in Fußnote

\textbf{Anmerkungen}

\begin{itemize}% Liste Punkt
\item
  Informationen, die den Lesefluss stören
\item
  Zusätzliche Erläuterungen
\item
  Hinweise zu weiterführender Literatur
\item
  Ausführliche Zitate
\item
  Anmerkungen stehen in Fuß- oder Endnoten
\end{itemize}

\subsubsection{Zitationsstile: Wie müssen Literaturangaben
aussehen?}\label{zitationsstile-wie-muessen-literaturangaben-aussehen}

\begin{itemize}% Liste Punkt
\item
  Deutschland: DIN ISO 690
\item
  APA Style
\item
  Chicago Style
\item
  Harvard Style
\item
  MLA Style
\end{itemize}

\subsubsection{Literaturangaben: Korrekte Quellennachweise für jede
Literaturart}\label{literaturangaben-korrekte-quellennachweise-fuer-jede-literaturart}

\textbf{Bibliographische Angaben}

\begin{itemize}% Liste Punkt
\item
  Name des Autors und Herausgebers
\item
  Titel und Untertitel, Titel des Beitrags
\item
  Nmae des Mediums/ der Zeitschrift
\item
  Verlag, Erscheinungsort, Auflage, Jahr
\item
  Seitenzahl
\item
  Zeitschriften: Jahrgang, laufende Nummer, Jahr
\end{itemize}

\textbf{Konventionen}

\begin{itemize}% Liste Punkt
\item
  Titel selbstständiger Werke werden kursiv, unselbstständige Beiträge
  in Anführungszeichen gesetzt
\item
  Bis zu drei Autoren/Herausgeber vollständig erfasst, bei mehreren nur
  der erste Name, gefolgt von \frqq u.a.\flqq\,
\item
  Abbkürzungen: a.a.O., Aufl., duchges., Diss., ebd., erw., f. und ff.
\end{itemize}

\textbf{Angaben von Online-Quellen}

\begin{itemize}% Liste Punkt
\item
  Wiedergabe der URL: (Online unter {[}kompletter Link{]})
\item
  Datum des Zugriffs (Zugriff am \ldots{})
\item
  Internetquelle nur bei Werken nennen, die ausschließlich online
  verfügbar sind
\item
  Download speichern und archivieren
\end{itemize}

\subsection{Abbildungen, Tabellen und
Diagramme}\label{abbildungen-tabellen-und-diagramme}

\subsubsection{\texorpdfstring{Wozu Abbildungen? - Aussagen \frqq auf den
ersten Blick\flqq\,
vermitteln}{Wozu Abbildungen? - Aussagen auf den ersten Blick vermitteln}}\label{wozu-abbildungen-aussagen-auf-den-ersten-blick-vermitteln}

\textbf{Bildunterschriften}

\begin{itemize}% Liste Punkt
\item
  Bildunterschriften formulieren die wesentliche Aussage
\end{itemize}

\textbf{Bildrechte respektieren}

\begin{itemize}% Liste Punkt
\item
  Bilder können im Rahmen des Zitatsrechts in die Arbeit eingefügt
  werden
\item
  Obligatorisch: genauer Quellennachweis, auch mit angabe des Copyrights
\item
  Auch bei Abbildungen aus dem Internet muss Quelle genannt werden
\end{itemize}

\subsubsection{Tabellen und Diagramme: Anschauliche Darstellung von
Zahlen}\label{tabellen-und-diagramme-anschauliche-darstellung-von-zahlen}

\textbf{Zahlen im Zusammenhang sehen}

\begin{itemize}% Liste Punkt
\item
  Tabellen und Diagramme zeigen eine Entwicklung auf
\item
  Auf dieser Grundlage erfolgt die Interpretation des Zahlenmaterials im
  Text
\item
  Repräsentativer Ausschnitt lenkt den Blick auf die wesentlichen
  Ergebnisse
\item
  Quelle der Daten nennen
\end{itemize}

\textbf{Tabellenüberschriften}

\begin{itemize}% Liste Punkt
\item
  Tabellenüberschriften beschreiben den dargestellten Inhalt
\end{itemize}

\subsection{Tipps}\label{tipps}

\subsubsection{\texorpdfstring{Schreiben mit \frqq System\flqq\, - und Hilfe im
Notfall}{Schreiben mit System - und Hilfe im Notfall}}\label{schreiben-mit-system-und-hilfe-im-notfall}

\begin{itemize}% Liste Punkt
\item
  Schreibportale: z.B.
  \url{http://www.studierendenakademie.hhu.de/schreibportal.html}
\item
  Korrekturleser gesucht!
\item
  Inhaltlich: logischer Aufbau der Gliederung, Richtigkeit aller
  Angaben. Redundanzen
\item
  Formal: Umsetzung der Vorgabe, Vollständigkeit aller Quellenangaben,
  einheitliche Gestaltung
\item
  Rechtschreibung und Terminologie
\end{itemize}

\subsection{Checklisten}\label{checklisten}

\subsubsection{Inhaltliche und organisatorische Vorbereitungen}\label{inhaltliche-und-organisatorische-vorbereitungen}

\begin{itemize}% Liste Punkt
\item Themensuche
\item Konzept erstellen
\item Literatursuche
\item Anlegen einer Stoffsammlung
\item Planen und durchführen empirischer Untersuchungen
\item Gliederungsentwurf
\item Ausgangsthese
\item Formalien und technische Ausstattung prüfen 
\end{itemize}


\subsubsection{Vorbereitungen für das Schreiben}\label{vorbereitungen-fuer-das-schreiben}

\begin{itemize}% Liste Punkt
\item Notwendig 
\item Auswahl und Exzerpieren der Quellen 
\item Auswerten empirischer Daten 
\item Gliederung erstellen 
\item Zusätzl. Elemente wie Abb. notwendig? 
\end{itemize}

\subsubsection{Ausarbeitung und Endkorrektur}\label{ausarbeitung-und-endkorrektur}

\begin{itemize}% Liste Punkt
\item Ausformulieren  
\item Anpassen der Gliederung  
\item Inhaltliche Endkorrektur: Log. Aufbau der Gliederung 
\item Richtigkeit aller Daten und Angaben 
\item Keine Redundanzen 
\item Wissenschaftlicher Stil eingehalten 
\item Formale Endkorrektur: Formale Vorgaben 
\item Literaturangaben gem. Zitationsstil 
\item Vollständigkeit aller Quellenangaben 
\item Einheitliche Gestaltung 
\item Alle Verzeichnisse erstellt 
\item Rechtschreibung 
\item Terminologie 
\end{itemize}
%============================
%============================
%============================
%============================
%============================
