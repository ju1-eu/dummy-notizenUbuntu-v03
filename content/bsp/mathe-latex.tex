%============================
% Copyright Jan Unger, Wuppertal
% erstellt: 22.12.16
% update:   23.06.17
%============================

%============================
% \chapter{Kapitel}
%============================

% 9.6.17

\section{Mathe - LaTeX}\label{matheLatex}

\lstset{language=[LaTeX]TeX } % ohne Zeilennummern, Hindergrund: weiss, ohne Rahmen
\begin{lstlisting}[gobble=2, frame=none, numbers=none, backgroundcolor=\color{white},%
	caption={},label={code:}]
  \begin{align}

  \end{align}
\end{lstlisting}

\subsection{Bruch}\label{bruch}\index{Bruch}

\begin{align}
	\frac{1}{3}
\end{align}

\lstset{language=[LaTeX]TeX } % ohne Zeilennummern, Hindergrund: weiss, ohne Rahmen
\begin{lstlisting}[gobble=2, frame=none, numbers=none, backgroundcolor=\color{white},%
	caption={},label={code:}]
  \frac{1}{3}
\end{lstlisting}


\subsection{Indizes}\label{indizes}\index{Indizes}

\begin{align}
  a_1
\end{align}

\lstset{language=[LaTeX]TeX } % ohne Zeilennummern, Hindergrund: weiss, ohne Rahmen
\begin{lstlisting}[gobble=2, frame=none, numbers=none, backgroundcolor=\color{white},%
	caption={},label={code:}]
  a_1
\end{lstlisting}

\subsection{Hochstellungen}\label{hochstellungen}\index{Hochstellungen}

\begin{align}
  a^2
\end{align}

\lstset{language=[LaTeX]TeX } % ohne Zeilennummern, Hindergrund: weiss, ohne Rahmen
\begin{lstlisting}[gobble=2, frame=none, numbers=none, backgroundcolor=\color{white},%
	caption={},label={code:}]
	a^2
\end{lstlisting}


\subsection{Summe}\label{summe}\index{Summe}

\begin{align}
  \sum \limits_{i=1}^n a_i
		= a_1 + a_2 + a_3 + ... + a_n
		= \frac{n(n+1)}{2}
\end{align}

\lstset{language=[LaTeX]TeX } % ohne Zeilennummern, Hindergrund: weiss, ohne Rahmen
\begin{lstlisting}[gobble=2, frame=none, numbers=none, backgroundcolor=\color{white},%
	caption={},label={code:}]
	\sum \limits_{i=1}^n a_i
			= a_1 + a_2 + a_3 + ... + a_n
			= \frac{n(n+1)}{2}
\end{lstlisting}


\subsection{Produkte}\label{produkte}\index{Produkte}

\begin{align}
  a \cdot b \quad \text{oder} \quad
  a \times b
\end{align}

\lstset{language=[LaTeX]TeX } % ohne Zeilennummern, Hindergrund: weiss, ohne Rahmen
\begin{lstlisting}[gobble=2, frame=none, numbers=none, backgroundcolor=\color{white},%
	caption={},label={code:}]
	a \cdot b \quad \text{oder} \quad
  a \times b
\end{lstlisting}


\subsection{Binome}\label{binome}\index{Binome}

\begin{align}
  & a \choose b \\
  & \binom{n}{k}
\end{align}

\lstset{language=[LaTeX]TeX } % ohne Zeilennummern, Hindergrund: weiss, ohne Rahmen
\begin{lstlisting}[gobble=2, frame=none, numbers=none, backgroundcolor=\color{white},%
	caption={},label={code:}]
	& a \choose b \\
  & \binom{n}{k}
\end{lstlisting}


\subsection{Wurzeln}\label{wurzeln}\index{Wurzeln}

\begin{align}
  \sqrt{x} \quad \text{oder} \quad
  \sqrt[n]{x}
\end{align}

\lstset{language=[LaTeX]TeX } % ohne Zeilennummern, Hindergrund: weiss, ohne Rahmen
\begin{lstlisting}[gobble=2, frame=none, numbers=none, backgroundcolor=\color{white},%
	caption={},label={code:}]
	\sqrt{x} \quad \text{oder} \quad
  \sqrt[n]{x}
\end{lstlisting}


\subsection{Limes}\label{limes}\index{Limes}

\begin{align}
  \lim\limits_{n \to \infty}\frac{1}{n}=0
\end{align}

\lstset{language=[LaTeX]TeX } % ohne Zeilennummern, Hindergrund: weiss, ohne Rahmen
\begin{lstlisting}[gobble=2, frame=none, numbers=none, backgroundcolor=\color{white},%
	caption={},label={code:}]
	\lim\limits_{n \to \infty}\frac{1}{n}=0
\end{lstlisting}


\subsection{Sonderzeichen}\label{sonderzeichen}\index{Sonderzeichen}

\begin{align}
  1 \le 2 \ge 0 \neq 4 \\
  1 \ll 10^{20} \gg 10^{-5} \\
  \pm 5
\end{align}

\lstset{language=[LaTeX]TeX } % ohne Zeilennummern, Hindergrund: weiss, ohne Rahmen
\begin{lstlisting}[gobble=2, frame=none, numbers=none, backgroundcolor=\color{white},%
	caption={},label={code:}]
	1 \le 2 \ge 0 \neq 4 \\
  1 \ll 10^{20} \gg 10^{-5} \\
  \pm 5
\end{lstlisting}


\subsection{Formel}\label{formel }\index{Formel }

Formel $P = \frac{W}{t}$ im Satz.

\begin{align}% Text: \texttt{Mathe-Text}
	Leistung &= Spannung \cdot Strom 	&& \left // \texttt{bündig Gleichheitszeichen} \right. \\
	       P &= U \cdot I 										\\
	   [ W ] &= [ V ] \cdot [ A ]
\end{align}

\lstset{language=[LaTeX]TeX } % ohne Zeilennummern, Hindergrund: weiss, ohne Rahmen
\begin{lstlisting}[gobble=2, frame=none, numbers=none, backgroundcolor=\color{white},%
	caption={},label={code:}]
	Formel $P = \frac{W}{t}$ im Satz.

	Leistung &= Spannung \cdot Strom 	&& \left // \texttt{bündig Gleichheitszeichen} \right. \\
	       P &= U \cdot I 										\\
	   [ W ] &= [ V ] \cdot [ A ]
\end{lstlisting}


\begin{align}% Text: \texttt{Mathe-Text}
& Leistung = Spannung \cdot Strom && \left // \texttt{linksbündig} \right. \\
& P = U \cdot I 										\\
& [ W ] = [ V ] \cdot [ A ]
\end{align}

\lstset{language=[LaTeX]TeX } % ohne Zeilennummern, Hindergrund: weiss, ohne Rahmen
\begin{lstlisting}[gobble=2, frame=none, numbers=none, backgroundcolor=\color{white},%
	caption={},label={code:}]
	& Leistung = Spannung \cdot Strom && \left // \texttt{linksbündig} \right. \\
	& P = U \cdot I 										\\
	& [ W ] = [ V ] \cdot [ A ]
\end{lstlisting}


\begin{align}% Text: \texttt{Mathe-Text}
x_{1,2}=\frac{-b \pm \sqrt{b^2-4\cdot a \cdot c}}{2 \cdot a}
\end{align}

\lstset{language=[LaTeX]TeX } % ohne Zeilennummern, Hindergrund: weiss, ohne Rahmen
\begin{lstlisting}[gobble=2, frame=none, numbers=none, backgroundcolor=\color{white},%
	caption={},label={code:}]
	x_{1,2}=\frac{-b \pm \sqrt{b^2-4\cdot a \cdot c}}{2 \cdot a}
\end{lstlisting}


\subsection{Relationen und Mengen}\label{RelationenMengen }\index{Relationen }\index{Mengen }

\begin{align}
<			      \quad
\leq 				\quad
>						\quad
\geq 				\quad
=						\quad
\subseteq 	\quad
\supseteq 	\quad
\in 				\quad
\notin 			\quad
\cap 				\quad
\cup        \quad
\mathbb{C \quad R \quad Q \quad Z \quad N}
\end{align}

\lstset{language=[LaTeX]TeX } % ohne Zeilennummern, Hindergrund: weiss, ohne Rahmen
\begin{lstlisting}[gobble=2, frame=none, numbers=none, backgroundcolor=\color{white},%
	caption={},label={code:}]
	<			      \quad
	\leq 				\quad
	>						\quad
	\geq 				\quad
	=						\quad
	\subseteq 	\quad
	\supseteq 	\quad
	\in 				\quad
	\notin 			\quad
	\cap 				\quad
	\cup        \quad
	\mathbb{C \quad R \quad Q \quad Z \quad N}
\end{lstlisting}


\subsection{Funktionen}\label{funktionen }\index{Funktionen }

\begin{align}
\exp
\lim
\ln
\log
\sin
\cos
\tan
\end{align}

\lstset{language=[LaTeX]TeX } % ohne Zeilennummern, Hindergrund: weiss, ohne Rahmen
\begin{lstlisting}[gobble=2, frame=none, numbers=none, backgroundcolor=\color{white},%
	caption={},label={code:}]
	\exp
	\lim
	\ln
	\log
	\sin
	\cos
	\tan
\end{lstlisting}


\subsection{Einheiten und Winkelgrad}\label{einheiten }\index{Einheiten }

\begin{align}
100\,^{\circ}\mathrm{C} \quad
360^\circ               \quad
1,8\,\Omega             \quad
500\,ml.
\end{align}

\lstset{language=[LaTeX]TeX } % ohne Zeilennummern, Hindergrund: weiss, ohne Rahmen
\begin{lstlisting}[gobble=2, frame=none, numbers=none, backgroundcolor=\color{white},%
	caption={},label={code:}]
	100\,^{\circ}\mathrm{C} \quad
	360^\circ               \quad
	1,8\,\Omega             \quad
	500\,ml.
\end{lstlisting}


\subsection{griechische Alphabet}\label{griechischeAlphabet }\index{griechische Alphabet }

\begin{align}
\alpha
\beta
\gamma
\delta
\epsilon
\varepsilon
\zeta
\eta
\pi
\varpi
\rho
\varrho
\sigma
\varsigma
\tau
\upsilon
\phi
\varphi
\chi
\psi
\omega \quad \texttt{klein}
\end{align}

\lstset{language=[LaTeX]TeX } % ohne Zeilennummern, Hindergrund: weiss, ohne Rahmen
\begin{lstlisting}[gobble=2, frame=none, numbers=none, backgroundcolor=\color{white},%
	caption={},label={code:}]
	\alpha
	\beta
	\gamma
	\delta
	\epsilon
	\varepsilon
	\zeta
	\eta
	\pi
	\varpi
	\rho
	\varrho
	\sigma
	\varsigma
	\tau
	\upsilon
	\phi
	\varphi
	\chi
	\psi
	\omega \quad \texttt{klein}
\end{lstlisting}


\begin{align}
A
E
\Gamma
\Delta
\Theta
\Lambda
\Xi
\Pi
\Sigma
\Upsilon
\Phi \quad \texttt{groß}
\end{align}

\lstset{language=[LaTeX]TeX } % ohne Zeilennummern, Hindergrund: weiss, ohne Rahmen
\begin{lstlisting}[gobble=2, frame=none, numbers=none, backgroundcolor=\color{white},%
	caption={},label={code:}]
	A
	E
	\Gamma
	\Delta
	\Theta
	\Lambda
	\Xi
	\Pi
	\Sigma
	\Upsilon
	\Phi \quad \texttt{groß}
\end{lstlisting}


\subsection{Fallunterscheidung}\label{fallunterscheidung }\index{Fallunterscheidung }

\begin{align}
\texttt{sin}(x) =
	\begin{cases}
		-1 & \texttt{für } x<0 \\
		 0 & \texttt{für } x=0 \\
		 1 & \texttt{für } x>0
	\end{cases}
\end{align}

\lstset{language=[LaTeX]TeX } % ohne Zeilennummern, Hindergrund: weiss, ohne Rahmen
\begin{lstlisting}[gobble=2, frame=none, numbers=none, backgroundcolor=\color{white},%
	caption={},label={code:}]
	\texttt{sin}(x) =
		\begin{cases}
			-1 & \texttt{für } x<0 \\
			 0 & \texttt{für } x=0 \\
			 1 & \texttt{für } x>0
		\end{cases}
\end{lstlisting}


\subsection{Punkte}\label{punkte }\index{Punkte }

\begin{align}
& x_1,x_2,\ldots{},x_n     \\		% linedots = zwischen variablen
& x_1 + x_2 + \cdots + x_n \\		% centerdots = zwischen operatoren
\end{align}

\lstset{language=[LaTeX]TeX } % ohne Zeilennummern, Hindergrund: weiss, ohne Rahmen
\begin{lstlisting}[gobble=2, frame=none, numbers=none, backgroundcolor=\color{white},%
	caption={},label={code:}]
	& x_1,x_2,\ldots{},x_n     \\		% linedots = zwischen variablen
	& x_1 + x_2 + \cdots + x_n  		% centerdots = zwischen operatoren
\end{lstlisting}


\subsection{Klammern}\label{klammern }\index{Klammern }

\begin{align}
& ((a+b) \cdot (a+b))\\
& \bigl((a+b) \cdot (a+b)\bigr)\\
& \left(\sum_{i=1}^5\right)\\
& \left\lvert \sum_{i=1}^5 \right\rvert
\end{align}

\lstset{language=[LaTeX]TeX } % ohne Zeilennummern, Hindergrund: weiss, ohne Rahmen
\begin{lstlisting}[gobble=2, frame=none, numbers=none, backgroundcolor=\color{white},%
	caption={},label={code:}]
	& ((a+b) \cdot (a+b))\\
	& \bigl((a+b) \cdot (a+b)\bigr)\\
	& \left(\sum_{i=1}^5\right)\\
	& \left\lvert \sum_{i=1}^5 \right\rvert
\end{lstlisting}


\subsection{Abstände}\label{abstaende }\index{Abstände }

\begin{align}
& x>y \quad  \text{, Abstand } \\
& x>y \qquad \text{, Abstand } \\
& x>y \,     \text{, Abstand }
\end{align}

\lstset{language=[LaTeX]TeX } % ohne Zeilennummern, Hindergrund: weiss, ohne Rahmen
\begin{lstlisting}[gobble=2, frame=none, numbers=none, backgroundcolor=\color{white},%
	caption={},label={code:}]
	& x>y \quad  \text{, Abstand } \\
	& x>y \qquad \text{, Abstand } \\
	& x>y \,     \text{, Abstand }
\end{lstlisting}


\subsection{Vektoren}\label{vektoren }\index{Vektoren }

\begin{align}
\vec{a}          \\   % vektor
\overrightarrow{a+b}  % vektoraddition
\end{align}

\lstset{language=[LaTeX]TeX } % ohne Zeilennummern, Hindergrund: weiss, ohne Rahmen
\begin{lstlisting}[gobble=2, frame=none, numbers=none, backgroundcolor=\color{white},%
	caption={},label={code:}]
	\vec{a}          \\   % vektor
	\overrightarrow{a+b}  % vektoraddition
\end{lstlisting}


\begin{align}
A = \begin{pmatrix} % runde Klammer
			a_{1} \\
			a_{2}
	  \end{pmatrix}
\end{align}

\lstset{language=[LaTeX]TeX } % ohne Zeilennummern, Hindergrund: weiss, ohne Rahmen
\begin{lstlisting}[gobble=2, frame=none, numbers=none, backgroundcolor=\color{white},%
	caption={},label={code:}]
	A = \begin{pmatrix} % runde Klammer
				a_{1} \\
				a_{2}
		  \end{pmatrix}
\end{lstlisting}


\subsection{Matrizen}\label{matrizen}\index{Matrizen}

\begin{align}
  \left(
   \begin{array}{ccc}
     a_{11} & \cdots & a_{1n} \\
     \vdots & \ddots & \vdots \\
     a_{m1} & \cdots & a_{mn}
   \end{array}
  \right)
\end{align}

\lstset{language=[LaTeX]TeX } % ohne Zeilennummern, Hindergrund: weiss, ohne Rahmen
\begin{lstlisting}[gobble=2, frame=none, numbers=none, backgroundcolor=\color{white},%
	caption={},label={code:}]
	\left(
   \begin{array}{ccc}
     a_{11} & \cdots & a_{1n} \\
     \vdots & \ddots & \vdots \\
     a_{m1} & \cdots & a_{mn}
   \end{array}
  \right)
\end{lstlisting}

\begin{align}
A = \begin{pmatrix} % runde Klammer
			a_{11} & a_{12} & \hdots \\
			a_{21} & a_{22} & \hdots \\
			\vdots & \vdots 	& \ddots
		\end{pmatrix}
\end{align}


\lstset{language=[LaTeX]TeX } % ohne Zeilennummern, Hindergrund: weiss, ohne Rahmen
\begin{lstlisting}[gobble=2, frame=none, numbers=none, backgroundcolor=\color{white},%
	caption={},label={code:}]
	A = \begin{pmatrix} % runde Klammer
				a_{11} & a_{12} & \hdots \\
				a_{21} & a_{22} & \hdots \\
				\vdots & \vdots 	& \ddots
			\end{pmatrix}
\end{lstlisting}


\begin{align}
A = \begin{vmatrix} % vertikaler Strich
			a_{11} & a_{12} & \hdots \\
			a_{21} & a_{22} & \hdots \\
			\vdots & \vdots 	& \ddots
		\end{vmatrix}
\end{align}

\lstset{language=[LaTeX]TeX } % ohne Zeilennummern, Hindergrund: weiss, ohne Rahmen
\begin{lstlisting}[gobble=2, frame=none, numbers=none, backgroundcolor=\color{white},%
	caption={},label={code:}]
	A = \begin{vmatrix} % vertikaler Strich
				a_{11} & a_{12} & \hdots \\
				a_{21} & a_{22} & \hdots \\
				\vdots & \vdots 	& \ddots
			\end{vmatrix}
\end{lstlisting}


\subsection{Einheiten}\label{einheiten }\index{Einheiten }

\num{12345678}\\
\num{.12345678}\\

\num{2.3e2}\\
\num{2.3d-2}\\

\lstset{language=[LaTeX]TeX } % ohne Zeilennummern, Hindergrund: weiss, ohne Rahmen
\begin{lstlisting}[gobble=2, frame=none, numbers=none, backgroundcolor=\color{white},%
	caption={},label={code:}]
	\num{12345678}\\
	\num{.12345678}\\

	\num{2.3e2}\\
	\num{2.3d-2}
\end{lstlisting}


\ang{13;12;11}\\

\lstset{language=[LaTeX]TeX } % ohne Zeilennummern, Hindergrund: weiss, ohne Rahmen
\begin{lstlisting}[gobble=2, frame=none, numbers=none, backgroundcolor=\color{white},%
	caption={},label={code:}]
	\ang{13;12;11}
\end{lstlisting}


\sisetup{per-mode=symbol}
\SI{1.7e2}{\pico\joule\per\kilo\gram\squared}\\

\lstset{language=[LaTeX]TeX } % ohne Zeilennummern, Hindergrund: weiss, ohne Rahmen
\begin{lstlisting}[gobble=2, frame=none, numbers=none, backgroundcolor=\color{white},%
	caption={},label={code:}]
	\sisetup{per-mode=symbol}
	\SI{1.7e2}{\pico\joule\per\kilo\gram\squared}
\end{lstlisting}


\sisetup{per-mode=fraction}
\SI{1.7e2}{\pico\joule\per\kilo\gram\squared}\\

\lstset{language=[LaTeX]TeX } % ohne Zeilennummern, Hindergrund: weiss, ohne Rahmen
\begin{lstlisting}[gobble=2, frame=none, numbers=none, backgroundcolor=\color{white},%
	caption={},label={code:}]
	\sisetup{per-mode=fraction}
	\SI{1.7e2}{\pico\joule\per\kilo\gram\squared}\\
\end{lstlisting}


\SI{2.8}{\meter\cubed}\\
\SI{2.8}{\meter\tothe{5}}\\
\SI{2 x 3 x 4}{\milli\meter}\\
\SI[product-units=power]{2 x 3 x 4}{\milli\meter}\\


\lstset{language=[LaTeX]TeX } % ohne Zeilennummern, Hindergrund: weiss, ohne Rahmen
\begin{lstlisting}[gobble=2, frame=none, numbers=none, backgroundcolor=\color{white},%
	caption={},label={code:}]
	\SI{2.8}{\meter\cubed}\\
	\SI{2.8}{\meter\tothe{5}}\\
	\SI{2 x 3 x 4}{\milli\meter}\\
	\SI[product-units=power]{2 x 3 x 4}{\milli\meter}
\end{lstlisting}


%============================
%============================
%============================
%============================
%============================
