%============================
% Copyright Jan Unger, Wuppertal
% erstellt: 22.12.16
% update:   23.06.17
%============================

%============================
 %\chapter{Formelsammlung}
%============================

% 9.6.17

\section{Mathe - Grundlagen}\label{matheGrundlagen}

% \subsection{ }\label{ }\index{ }

% \begin{align}	\quad \text{;} \quad \end{align}


\subsection{Potenzen }\label{potenzen }\index{Potenzen }

allgemein:

\begin{align}
a^n &= a \cdot a \cdot ... \cdot a_n
\end{align}

Multiplikation: (gl.Basis, gl. Exponent)

\begin{align}
a^n \cdot a^m &= a^{n+m} \\
a^n \cdot b^n &= (a \cdot b)^n
\end{align}

Division:

\begin{align}
\frac{a^n}{a^m} &= a^{n-m} \\
a^{-n}          &= \frac{1}{a^n}   \\
a^0 						&= 1 \\
a^1				      &= a
\end{align}

Potenzen potenzieren:

\begin{align}
(a^n)^m         &= a^{n \cdot m} \\
\frac{a^n}{b^n} &= \left(\frac{a}{b}\right)^n
\end{align}

\begin{align}
a^b        &= e^{b \cdot ln \, a} \\
\sqrt[n]{a^m} &= a^{\frac{m}{n}}
\end{align}
%============================
%============================
%============================
%============================
%============================
