% letztes Update: 28-Dez-19
\section{Markdown - Spickzettel}\label{markdown-spickzettel}

\subsection{Quellenangabe}\label{quellenangabe}

Zitat: vgl. \cite{monk_action_buch:2016} u. \cite{kofler_linux:2017}

\subsection{Listen}\label{listen}

\textbf{ungeordnete Liste}

\begin{itemize}% Liste Punkt
\item
  a
\item
  b

  \begin{itemize}% Liste Punkt
  \item
    bb
  \end{itemize}
\item
  c
\end{itemize}

\textbf{Sortierte Liste}

\begin{enumerate}% Liste 1) oder a)
\item
  eins
\item
  zwei
\item
  drei
\end{enumerate}

\textbf{Sortierte Liste}

\begin{enumerate}% Liste 1) oder a)
\def\labelenumi{\alph{enumi})}
\item
  a
\item
  b
\item
  c
\end{enumerate}

\subsection{Anführungszeichen}\label{anfuehrungszeichen}

<<Anführungszeichen>>

\subsection{Bilder - Abbildungen}\label{bilder-abbildungen}

%Bild (siehe Abb. \ref{fig:}). 
\begin{figure}[H]% hier: htbp
	\centering
	\includegraphics[width=.55\textwidth]{images/logo.pdf}
	%+------------------------
	%\caption{}%\label{fig:}% 
\end{figure}

\subsection{Tabelle}\label{tabelle}

 %Tabelle (vgl. Tab. \ref{tab:}). 
	\begin{table}[H]% hier: htbp 
	\centering 
	\arrayrulecolor{lightgray} 
	\sffamily\footnotesize 
	\begin{tabular}{@{}rll@{}}
	\toprule 
\textbf{Nr.} & \textbf{Begriffe} & \textbf{Erklärung}\\
	\midrule
1 & a1 & a2\\
2 & b1 & b2\\
3 & c1 & c2\\
4 & a1 & a2\\
	\bottomrule
 \end{tabular}%
	%+------------------------
	%\caption{}%\label{tab:}% 
\end{table}

\subsection{Mathe}\label{mathe}

$[ V ] = [ \Omega ] \cdot [ A ]$ o. $U = R \cdot I$ o.
$R = \frac{U}{I}$

\textbf{Matheumgebung:}

\begin{align*}
    \sum_{i=1}^5 a_i = a_1 + a_2 + a_3 + a_4 + a_5
\end{align*}

\subsection{Texthervorhebung}\label{texthervorhebung}

\textbf{Fett} oder \emph{Kursiv}

\subsection{Code}\label{code}

%Quellcode (vgl. Quellcode \ref{code:}). 
\begin{lstlisting}[language=Bash,% C, C++, TeX, Bash, Python 
%+------------------------
%caption={Quellcode in Bash: },%label={code:}% 
]%--Code einfügen--% 

    #include <stdio.h>
    int main(void) {
        printf("Hallo Welt!\n");
        return 0;
    }
\end{lstlisting}

\subsection{Links}\label{links}

\url{https://google.de} oder \href{https://google.de}{Google}

\subsection{Absätze}\label{absaetze}

Dies hier ist ein Blindtext zum Testen von Textausgaben. Wer diesen Text
liest, ist selbst schuld. Der Text gibt lediglich den Grauwert der
Schrift an. Ist das wirklich so? Ist es gleichgültig, ob ich schreibe:
<<Dies ist ein Blindtext>> oder <<Huardest gefburn>>? Kjift -
mitnichten! Ein Blindtext bietet mir wichtige Informationen.

Dies hier ist ein Blindtext zum Testen von Textausgaben. Wer diesen Text
liest, ist selbst schuld. Der Text gibt lediglich den Grauwert der
Schrift an. Ist das wirklich so? Ist es gleichgültig, ob ich schreibe:
<<Dies ist ein Blindtext>> oder <<Huardest gefburn>>? Kjift -
mitnichten! Ein Blindtext bietet mir wichtige Informationen.
